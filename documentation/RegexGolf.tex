\documentclass{article}
\usepackage[utf8]{inputenc}
\usepackage[serbian]{babel} 
\usepackage{listings}
\usepackage{graphicx}
\usepackage{hyperref}
\hypersetup{
colorlinks,
linkcolor=blue,
urlcolor=blue
}
\setlength{\textheight}{600pt}
\setlength{\textwidth}{140mm}
\setlength{\topmargin}{5pt}
\setlength{\evensidemargin}{53pt}
\setlength{\oddsidemargin}{10mm}

\title{%
  Playing Regex Golf with Genetic Programming \vspace{0.4cm} \\ 
  \large Projekat u okviru kursa Računarska inteligencija \\
  Matematički fakultet\\ Univerzitet u Beogradu \vspace*{0.5cm}}
  
\author{Anđela Ilić \\
\href{mailto:mi17105@alas.matf.bg.ac.rs}{mi17105@alas.matf.bg.ac.rs} \\
Mina Milošević \\
\href{mailto:mi17081@alas.matf.bg.ac.rs}{mi17081@alas.matf.bg.ac.rs} \\
}

\date{\vspace*{1cm}Februar 2021}

\begin{document}

\maketitle

\newpage

\renewcommand*\contentsname{Sadržaj}
\tableofcontents
\newpage

\section{Opis problema}
Data su dva skupa reči - M i U. Cilj \textit{Regex Golf} igre je pronaći najkraći regularni izraz kojim se mogu zapisati sve reči iz skupa M, ali kojim se ne može zapisati nijedna reč skupa U. Za date skupove M i U ne možemo sa sigurnošću da tvrdimo da postoji rešenje koje zadovoljava prethodne uslove. Takođe, ako dobijemo regularni izraz koji zadovoljava navedene uslove, ne možemo za svaki primer znati da li postoji i bolje rešenje tj. kraći regularni izraz. 

\section{Implementacija}
Svaka jedinka u Genetskom programiranju će biti predstavljena kao drvo.
U listovima nalaze elementi koje ćemo jednim imenom zvati \textit{Terminali} (terminal set), a u unutrašnjim čvorovima su elementi koje nazivamo \textit{Funkcije} (function set). \\
Skup funkcija sadrži operatore koji se mogu javiti u regularnim izrazima. Primeri takvih operatora su: $.*+$,, $.++$, $.?+$, $.{.,.}+$, $(.)$, $[.]$,
$[^.]$, $..$, $.|.$. Tačka $.$ je mesto na kome se nalaze deca u drvetu. \\
Skup terminala čine elementi koji zavise i koji ne zavise od ulaznih skupova M i U. Elementi koji su nezavisni - opsezi malih i velikih slova, brojeva u
regularnim izrazima, karakteri $\textasciicircum$ i $\$$, wildcard karakter '$..$'. Elementi skupa terminala koji su zavisni - skup karaktera iz M, opsezi karaktera iz M i n-grami.

\subsection{Priprema ulaznih podataka}



\subsection{Genetsko programiranje}
\subsubsection{Implementacija jedinki}
\subsubsection{Parametri genetskog programiranja}
%\subsubsection{Selekcija, ukrštanje, mutacija}

\section{Rezultati}
\section{Zaključak}
\section{Reference}


\end{document}