\section{Implementacija}
    
    \frame{\sectionpage}
    
    \begin{frame}{Terminal i Function skupovi}
        Svaka jedinka će biti predstavljena kao drvo. U listovima
        nalaze elementi koje ćemo jednim imenom zvati \textit{Terminali} (terminal set), a u unutrašnjim čvorovima su elementi koje nazivamo \textit{Funkcije} (function set).
    \end{frame}
    
    \begin{frame}{Terminal i Function skupovi}
        Skup funkcija sadrži operatore koji se mogu javiti u regularnim izrazima. Primeri takvih operatora su: $.*$, $.+$, $.?$, $.\{.,.\}+$, $(.)$, $[.]$, $[\ \textasciicircum .]$, $..$, $.|.$. Tačka . je mesto na kome se nalaze deca u drvetu. \\
        Skup terminala čine elementi koji zavise i koji ne zavise od ulaznih skupova M i U. Elementi
        koji su nezavisni - opsezi malih i velikih slova, brojeva u regularnim izrazima, karakteri $\textasciicircum$ i $\$$, wildcard karakter '$\%$'. Elementi skupa terminala koji su zavisni - skup karaktera iz M,
        parcijalni opsezi karaktera iz M i n-grami.
    \end{frame}
    
    \begin{frame}{Genetsko programiranje - jedinke}
        Svaka jedinka se predstavlja preko stabla. U korenu stabla se nalazi karakter ’.’ i koren uvek ima dva deteta. Elementi stabla se biraju nasumično iz  skupova \textit{Function} i \textit{Terminal}. \\
        Od kreiranog drveta se dobija niska koja predstavlja validan regularni izraz. \\
        Za svaku jedinku računamo i \textit{fitnes} funkciju po formuli:
        $$f(x) = w_i * (n_m - n_u) - length(r)$$
        gde je $w_i$ statistički određena konstanta, $n_m$ i $n_u$ brojevi reči iz skupova M i U, redom, koje su opisane dobijenim regularnim izrazom $r$. Ovako definisanu funkciju maksimizujemo.
    \end{frame}
    
    \begin{frame}{Genetsko programiranje - selekcija}
        Za \textit{selekciju} koristimo turnirsku selekciju veličine 7. Jedinke za selekciju biramo nasumično i uzimamo najbolju jedinku tj. onu koja ima najveći fitnes među odabranim.
    \end{frame}
    
    \begin{frame}{Genetsko programiranje - ukrštanje}
        Koristimo \textit{jednopoziciono ukrštanje}. \\
        Najpre obilazimo stabla roditelja BFS-om i numerišemo čvorove. \\
        Biramo jedan broj koji će predstavljati indeks čvora. Tražimo podstabla u roditeljima koja treba da razmene mesta. \\
        Na kraju treba proveriti da li su regularni izrazi ovako dobijene dece validni. U slučaju da neko dete nije validno, vraćamo odgovarajućeg roditelja umesto njega.
    \end{frame}
    
    \begin{frame}{Genetsko programiranje - mutacija}
        \textit{Mutaciju} radimo sa verovatnoćom 0.1. Biramo indeks čvora koji želimo da promenimo. Razlikujemo dva slučaja - kada je čvor iz skupa terminala i kada je iz skupa funkcija. \\
        U prvom slučaju, samo izaberemo nasumično novi terminal i ažuriramo vrednost čvora. \\
        Ako je čvor bio iz skupa funkcija, biramo novu vrednost iz istog skupa. Treba obratiti pažnju da li se kardinalnost (broj dece) čvora promenila.\\
        Na kraju opet treba proveriti da li je dobijena jedinka validna. Ako nije, vraćamo jedinku za koju smo i pokrenuli mutaciju. 
    \end{frame}
    
    \begin{frame}{Genetsko programiranje - parametri}
        Parametri su postavljeni na osnovu podataka iz \cite{Bartoli}:
        \begin{itemize}
            \item populaciju čini 500 jedinki
            \item pravimo 1000 generacija
            \item turnirska selekcija je dimenzije 7
            \item verovatnoća mutacije je 0.1
            \item elitizam iznosi $20\%$
            \item algoritam pozivamo za 30ak populacija
        \end{itemize}
    \end{frame}
    
    \begin{frame}{Izlaz programa}
        Iz svake populacije čuvamo sve jedinke kod kojih važi:
        $$n_m - n_u = |M|$$
        Dakle, želimo da sačuvamo sve jedinke kod kojih su ispunjena prva 2 uslova ovog problema. Ako u trenutnoj populaciji ne postoji takva jedinka, čuvamo samo najbolju jedinku na osnovu celokupnog fitnesa. \\
        Na kraju rada algoritma biramo najbolju jedinku, od svih sačuvanih, po njenom ukupnom fitnesu i nju proglasavamo za rešenjem datog problema.
    \end{frame}